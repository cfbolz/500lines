\begin{aosachapter}{DBDB: Dog Bed Database}{s:data-store}{Taavi Burns}

DBDB (Dog Bed Database) is a Python library that implements a simple
key/value database. It lets you associate a key with a value, and store
that association on disk for later retrieval.

DBDB aims to preserve data in the face of computer crashes and error
conditions. It also avoids holding all data in RAM at once so you can
store more data than you have RAM.

\aosasecti{Memory}\label{memory}

I remember the first time I was really stuck on a bug. When I finished
typing in my BASIC program and ran it, weird sparkly pixels showed up on
the screen, and the program aborted early. When I went back to look at
the code, the last few lines of the program were gone.

One of my mom's friends knew how to program, so we set up a call. Within
a few minutes of speaking with her, I found the problem: the program was
too big, and had encroached onto video memory. Clearing the screen
truncated the program, and the sparkles were artifacts of Applesoft
BASIC's behaviour of storing program state in RAM just beyond the end of
the program.

From that moment onwards, I cared about memory allocation. I learned
about pointers and how to allocate memory with malloc. I learned how my
data structures were laid out in memory. And I learned to be very, very
careful about how I changed them.

Some years later, while reading about a process-oriented language called
Erlang, I learned that it didn't actually have to copy data to send
messages between processes, because everything was immutable. I then
discovered immutable data structures in Clojure, and it really began to
sink in.

When I read about CouchDB in 2013, I just smiled and nodded, recognising
the structures and mechanisms for managing complex data as it changes.

I learned that you can design systems built around immutable data.

Then I agreed to write a book chapter.

I thought that describing the core data storage concepts of CouchDB (as
I understood them) would be fun.

While trying to write a binary tree algorithm that mutated the tree in
place, I got frustrated with how complicated things were getting. The
number of edge cases and trying to reason about how changes in one part
of the tree affected others was making my head hurt. I had no idea how I
was going to explain all of this.

Remembering lessons learned, I took a peek at a recursive algorithm for
updating immutable binary trees and it turned out to be relatively
straightforward.

I learned, once again, that it's easier to reason about things that
don't change.

So starts the story.

\aosasecti{Why Is it Interesting?}\label{why-is-it-interesting}

Most projects require a database of some kind. You really shouldn't
write your own; there are many edge cases that will bite you, even if
you're just writing JSON to disk:

\begin{aosaitemize}

\item
  What happens if your filesystem runs out of space?
\item
  What happens if your laptop battery dies while saving?
\item
  What if your data size exceeds available memory? (Unlikely for most
  applications on modern desktop computers\ldots{} but not unlikely for
  a mobile device or server-side web application.)
\end{aosaitemize}

However, if you want to \emph{understand} how a database handles all of
these problems, writing one for yourself can be a good idea.

The techniques and concepts we discuss here should be applicable to any
problem that needs to have rational, predictable behaviour when faced
with failure.

Speaking of failure\ldots{}

\aosasecti{Characterizing Failure}\label{characterizing-failure}

Databases are often characterized by how closely they adhere to the ACID
properties: atomicity, consistency, isolation, and durability.

Updates in DBDB are atomic and durable, two attributes which are
described later in the chapter. DBDB provides no consistency guarantees
as there are no constraints on the data stored. Isolation is likewise
not implemented.

Application code can, of course, impose its own consistency guarantees,
but proper isolation requires a transaction manager. We won't attempt
that here; however, you can learn more about transaction management in
the CircleDB chapter (\aosachapref{s:functionaldb}).

We also have other system-maintenance problems to think about. Stale
data is not reclaimed in this implementation, so repeated updates (even
to the same key) will eventually consume all disk space. (You will
shortly discover why this is the case.)
\href{http://www.postgresql.org/}{PostgreSQL} calls this reclamation
``vacuuming'' (which makes old row space available for re-use), and
\href{http://couchdb.apache.org/}{CouchDB} calls it ``compaction'' (by
rewriting the ``live'' parts of the data store into a new file, and
atomically moving it over the old one).

DBDB could be enhanced to add a compaction feature, but it is left as an
exercise for the reader\footnote{Bonus feature: Can you guarantee that
  the compacted tree structure is balanced? This helps maintain
  performance over time.}.

\aosasecti{The Architecture of DBDB}\label{the-architecture-of-dbdb}

DBDB separates the concerns of ``put this on disk somewhere'' (how data
are laid out in a file; the physical layer) from the logical structure
of the data (a binary tree in this example; the logical layer) from the
contents of the key/value store (the association of key \texttt{a} to
value \texttt{foo}; the public API).

Many databases separate the logical and physical aspects as it is is
often useful to provide alternative implementations of each to get
different performance characteristics, e.g.~DB2's SMS (files in a
filesystem) versus DMS (raw block device) tablespaces, or MySQL's
\href{http://dev.mysql.com/doc/refman/5.7/en/storage-engines.html}{alternative
engine implementations}.

\aosasecti{Discovering the Design}\label{discovering-the-design}

Most of the chapters in this book describe how a program was built from
inception to completion. However, that is not how most of us interact
with the code we're working on. We most often discover code that was
written by others, and figure out how to modify or extend it to do
something different.

In this chapter, we'll assume that DBDB is a completed project, and walk
through it to learn how it works. Let's explore the structure of the
entire project first.

\aosasectii{Organisational Units}\label{organisational-units}

Units are ordered here by distance from the end user; that is, the first
module is the one that a user of this program would likely need to know
the most about, while the last is something they should have very little
interaction with.

\begin{aosaitemize}
\item
  \texttt{tool.py} defines a command-line tool for exploring a database
  from a terminal window.
\item
  \texttt{interface.py} defines a class (\texttt{DBDB}) which implements
  the Python dictionary API using the concrete \texttt{BinaryTree}
  implementation. This is how you'd use DBDB inside a Python program.
\item
  \texttt{logical.py} defines the logical layer. It's an abstract
  interface to a key/value store.

  \begin{aosaitemize}
  \item
    \texttt{LogicalBase} provides the API for logical updates (like get,
    set, and commit) and defers to a concrete subclass to implement the
    updates themselves. It also manages storage locking and
    dereferencing internal nodes.
  \item
    \texttt{ValueRef} is a Python object that refers to a binary blob
    stored in the database. The indirection lets us avoid loading the
    entire data store into memory all at once.
  \end{aosaitemize}
\item
  \texttt{binary\_tree.py} defines a concrete binary tree algorithm
  underneath the logical interface.

  \begin{aosaitemize}
  \item
    \texttt{BinaryTree} provides a concrete implementation of a binary
    tree, with methods for getting, inserting, and deleting key/value
    pairs. \texttt{BinaryTree} represents an immutable tree; updates are
    performed by returning a new tree which shares common structure with
    the old one.
  \item
    \texttt{BinaryNode} implements a node in the binary tree.
  \item
    \texttt{BinaryNodeRef} is a specialised \texttt{ValueRef} which
    knows how to serialise and deserialise a \texttt{BinaryNode}.
  \end{aosaitemize}
\item
  \texttt{physical.py} defines the physical layer. The \texttt{Storage}
  class provides persistent, (mostly) append-only record storage.
\end{aosaitemize}

These modules grew from attempting to give each class a single
responsibility. In other words, each class should have only one reason
to change.

\aosasectii{Reading a Value}\label{reading-a-value}

We'll start with the simplest case: reading a value from the database.

Let's see what happens when we try to get the value associated with key
\texttt{foo} in \texttt{example.db}:

\begin{verbatim}
$ python -m dbdb.tool example.db get foo
\end{verbatim}

This runs the \texttt{main()} function from module \texttt{dbdb.tool}:

\begin{verbatim}
# dbdb/tool.py
def main(argv):
    if not (4 <= len(argv) <= 5):
        usage()
        return BAD_ARGS
    dbname, verb, key, value = (argv[1:] + [None])[:4]
    if verb not in {'get', 'set', 'delete'}:
        usage()
        return BAD_VERB
    db = dbdb.connect(dbname)          # CONNECT
    try:
        if verb == 'get':
            sys.stdout.write(db[key])  # GET VALUE
        elif verb == 'set':
            db[key] = value
            db.commit()
        else:
            del db[key]
            db.commit()
    except KeyError:
        print("Key not found", file=sys.stderr)
        return BAD_KEY
    return OK
\end{verbatim}

The \texttt{connect()} function opens the database file (possibly
creating it, but never overwriting it) and returns an instance of
\texttt{DBDB}:

\begin{verbatim}
# dbdb/__init__.py
def connect(dbname):
    try:
        f = open(dbname, 'r+b')
    except IOError:
        fd = os.open(dbname, os.O_RDWR | os.O_CREAT)
        f = os.fdopen(fd, 'r+b')
    return DBDB(f)
\end{verbatim}

\begin{verbatim}
# dbdb/interface.py
class DBDB(object):

    def __init__(self, f):
        self._storage = Storage(f)
        self._tree = BinaryTree(self._storage)
\end{verbatim}

We see right away that \texttt{DBDB} has a reference to an instance of
\texttt{Storage}, but it also shares that reference with
\texttt{self.\_tree}. Why? Can't \texttt{self.\_tree} completely manage
access to the storage by itself?

The question of which objects ``own'' a resource is often an important
one in a design, because it gives us hints about what changes might be
unsafe. Let's keep that question in mind as we move on.

Once we have a DBDB instance, getting the value at \texttt{key} is done
via a dictionary lookup (\texttt{db{[}key{]}}), which causes the Python
interpreter to call \texttt{DBDB.\_\_getitem\_\_()}.

\begin{verbatim}
# dbdb/interface.py
class DBDB(object):
# ...
    def __getitem__(self, key):
        self._assert_not_closed()
        return self._tree.get(key)

    def _assert_not_closed(self):
        if self._storage.closed:
            raise ValueError('Database closed.')
\end{verbatim}

\texttt{\_\_getitem\_\_()} ensures that the database is still open by
calling \texttt{\_assert\_not\_closed}. Aha! Here we see at least one
reason why \texttt{DBDB} needs direct access to our \texttt{Storage}
instance: so it can enforce preconditions. (Do you agree with this
design? Can you think of a different way that we could do this?)

DBDB then retrieves the value associated with \texttt{key} on the
internal \texttt{\_tree} by calling \texttt{\_tree.get()}, which is
provided by \texttt{LogicalBase}:

\begin{verbatim}
# dbdb/logical.py
class LogicalBase(object):
# ...
    def get(self, key):
        if not self._storage.locked:
            self._refresh_tree_ref()
        return self._get(self._follow(self._tree_ref), key)
\end{verbatim}

\texttt{get()} checks if we have the storage locked. We're not 100\%
sure \emph{why} there might be a lock here, but we can guess that it
probably exists to allow writers to serialize access to the data. What
happens if the storage isn't locked?

\begin{verbatim}
# dbdb/logical.py
class LogicalBase(object):
# ...
def _refresh_tree_ref(self):
        self._tree_ref = self.node_ref_class(
            address=self._storage.get_root_address())
\end{verbatim}

\texttt{\_refresh\_tree\_ref} resets the tree's ``view'' of the data
with what is currently on disk, allowing us to perform a completely
up-to-date read.

What if storage \emph{is} locked when we attempt a read? This means that
some other process is probably changing the data we want to read right
now; our read is not likely to be up-to-date with the current state of
the data. This is generally known as a ``dirty read''. This pattern
allows many readers to access data without ever worrying about blocking,
at the expense of being slightly out-of-date.

For now, let's take a look at how we actually retrieve the data:

\begin{verbatim}
# dbdb/binary_tree.py
class BinaryTree(LogicalBase):
# ...
    def _get(self, node, key):
        while node is not None:
            if key < node.key:
                node = self._follow(node.left_ref)
            elif node.key < key:
                node = self._follow(node.right_ref)
            else:
                return self._follow(node.value_ref)
        raise KeyError
\end{verbatim}

This is a standard binary tree search, following refs to their nodes. We
know from reading the \texttt{BinaryTree} documentation that
\texttt{Node}s and \texttt{NodeRef}s are value objects: they are
immutable and their contents never change. \texttt{Node}s are created
with an associated key and value, and left and right children. Those
associations also never change. The content of the whole
\texttt{BinaryTree} only visibly changes when the root node is replaced.
This means that we don't need to worry about the contents of our tree
being changed while we are performing the search.

Once the associated value is found, it is written to \texttt{stdout} by
\texttt{main()} without adding any extra newlines, to preserve the
user's data exactly.

\aosasectiii{Inserting and Updating}\label{inserting-and-updating}

Now we'll set key \texttt{foo} to value \texttt{bar} in
\texttt{example.db}:

\begin{verbatim}
$ python -m dbdb.tool example.db set foo bar
\end{verbatim}

Again, this runs the \texttt{main()} function from module
\texttt{dbdb.tool}. Since we've seen this code before, we'll just
highlight the important parts:

\begin{verbatim}
# dbdb/tool.py
def main(argv):
    ...
    db = dbdb.connect(dbname)          # CONNECT
    try:
        ...
        elif verb == 'set':
            db[key] = value            # SET VALUE
            db.commit()                # COMMIT
        ...
    except KeyError:
        ...
\end{verbatim}

This time we set the value with \texttt{db{[}key{]} = value} which calls
\texttt{DBDB.\_\_setitem\_\_()}.

\begin{verbatim}
# dbdb/interface.py
class DBDB(object):
# ...
    def __setitem__(self, key, value):
        self._assert_not_closed()
        return self._tree.set(key, value)
\end{verbatim}

\texttt{\_\_setitem\_\_} ensures that the database is still open and
then stores the association from \texttt{key} to \texttt{value} on the
internal \texttt{\_tree} by calling \texttt{\_tree.set()}.

\texttt{\_tree.set()} is provided by \texttt{LogicalBase}:

\begin{verbatim}
# dbdb/logical.py
class LogicalBase(object):
# ...
    def set(self, key, value):
        if self._storage.lock():
            self._refresh_tree_ref()
        self._tree_ref = self._insert(
            self._follow(self._tree_ref), key, self.value_ref_class(value))
\end{verbatim}

\texttt{set()} first checks the storage lock:

\begin{verbatim}
# dbdb/storage.py
class Storage(object):
    ...
    def lock(self):
        if not self.locked:
            portalocker.lock(self._f, portalocker.LOCK_EX)
            self.locked = True
            return True
        else:
            return False
\end{verbatim}

There are two important things to note here:

\begin{aosaitemize}

\item
  Our lock is provided by a 3rd-party file-locking library called
  \href{https://pypi.python.org/pypi/portalocker}{portalocker}.
\item
  \texttt{lock()} returns \texttt{False} if the database was already
  locked, and \texttt{True} otherwise.
\end{aosaitemize}

Returning to \texttt{\_tree.set()}, we can now understand why it checked
the return value of \texttt{lock()} in the first place: it lets us call
\texttt{\_refresh\_tree\_ref} for the most recent root node reference so
we don't lose updates that another process may have made since we last
refreshed the tree from disk. Then it replaces the root tree node with a
new tree containing the inserted (or updated) key/value.

Inserting or updating the tree doesn't mutate any nodes, because
\texttt{\_insert()} returns a new tree. The new tree shares unchanged
parts with the previous tree to save on memory and execution time. It's
natural to implement this recursively:

\begin{verbatim}
# dbdb/binary_tree.py
class BinaryTree(LogicalBase):
# ...
    def _insert(self, node, key, value_ref):
        if node is None:
            new_node = BinaryNode(
                self.node_ref_class(), key, value_ref, self.node_ref_class(), 1)
        elif key < node.key:
            new_node = BinaryNode.from_node(
                node,
                left_ref=self._insert(
                    self._follow(node.left_ref), key, value_ref))
        elif node.key < key:
            new_node = BinaryNode.from_node(
                node,
                right_ref=self._insert(
                    self._follow(node.right_ref), key, value_ref))
        else:
            new_node = BinaryNode.from_node(node, value_ref=value_ref)
        return self.node_ref_class(referent=new_node)
\end{verbatim}

Notice how we always return a new node (wrapped in a \texttt{NodeRef}).
Instead of updating a node to point to a new subtree, we make a new node
which shares the unchanged subtree. This is what makes this binary tree
an immutable data structure.

You may have noticed something strange here: we haven't made any changes
to anything on disk yet. All we've done is manipulate our view of the
on-disk data by moving tree nodes around.

In order to actually write these changes to disk, we need an explicit
call to \texttt{commit()}, which we saw as the second part of our
\texttt{set} operation in \texttt{tool.py} at the beginning of this
section.

Committing involves writing out all of the dirty state in memory, and
then saving the disk address of the tree's new root node.

Starting from the API:

\begin{verbatim}
# dbdb/interface.py
class DBDB(object):
# ...
    def commit(self):
        self._assert_not_closed()
        self._tree.commit()
\end{verbatim}

The implementation of \texttt{\_tree.commit()} comes from
\texttt{LogicalBase}:

\begin{verbatim}
# dbdb/logical.py
class LogicalBase(object)
# ...
    def commit(self):
        self._tree_ref.store(self._storage)
        self._storage.commit_root_address(self._tree_ref.address)
\end{verbatim}

All \texttt{NodeRef}s know how to serialise themselves to disk by first
asking their children to serialise via \texttt{prepare\_to\_store()}:

\begin{verbatim}
# dbdb/logical.py
class ValueRef(object):
# ...
    def store(self, storage):
        if self._referent is not None and not self._address:
            self.prepare_to_store(storage)
            self._address = storage.write(self.referent_to_string(self._referent))
\end{verbatim}

\texttt{self.\_tree\_ref} in \texttt{LogicalBase} is actually a
\texttt{BinaryNodeRef} (a subclass of \texttt{ValueRef}) in this case,
so the concrete implementation of \texttt{prepare\_to\_store()} is:

\begin{verbatim}
# dbdb/binary_tree.py
class BinaryNodeRef(ValueRef):
    def prepare_to_store(self, storage):
        if self._referent:
            self._referent.store_refs(storage)
\end{verbatim}

The \texttt{BinaryNode} in question, \texttt{\_referent}, asks its refs
to store themselves:

\begin{verbatim}
# dbdb/binary_tree.py
class BinaryNode(object):
# ...
    def store_refs(self, storage):
        self.value_ref.store(storage)
        self.left_ref.store(storage)
        self.right_ref.store(storage)
\end{verbatim}

This recurses all the way down for any \texttt{NodeRef} which has
unwritten changes (i.e., no \texttt{\_address}).

Now we're back up the stack in \texttt{ValueRef}'s \texttt{store} method
again. The last step of \texttt{store()} is to serialise this node and
save its storage address:

\begin{verbatim}
# dbdb/logical.py
class ValueRef(object):
# ...
    def store(self, storage):
        if self._referent is not None and not self._address:
            self.prepare_to_store(storage)
            self._address = storage.write(self.referent_to_string(self._referent))
\end{verbatim}

At this point the \texttt{NodeRef}'s \texttt{\_referent} is guaranteed
to have addresses available for all of its own refs, so we serialise it
by creating a bytestring representing this node:

\begin{verbatim}
# dbdb/binary_tree.py
class BinaryNodeRef(ValueRef):
# ...
    @staticmethod
    def referent_to_string(referent):
        return pickle.dumps({
            'left': referent.left_ref.address,
            'key': referent.key,
            'value': referent.value_ref.address,
            'right': referent.right_ref.address,
            'length': referent.length,
        })
\end{verbatim}

Updating the address in the \texttt{store()} method is technically a
mutation of the \texttt{ValueRef}. Because it has no effect on the
user-visible value, we can consider it to be immutable.

Once \texttt{store()} on the root \texttt{\_tree\_ref} is complete (in
\texttt{LogicalBase.commit()}), we know that all of the data are written
to disk. We can now commit the root address by calling:

\begin{verbatim}
# dbdb/physical.py
class Storage(object):
# ...
    def commit_root_address(self, root_address):
        self.lock()
        self._f.flush()
        self._seek_superblock()
        self._write_integer(root_address)
        self._f.flush()
        self.unlock()
\end{verbatim}

We ensure that the file handle is flushed (so that the OS knows we want
all the data saved to stable storage like an SSD) and write out the
address of the root node. We know this last write is atomic because we
store the disk address on a sector boundary. It's the very first thing
in the file, so this is true regardless of sector size, and
single-sector disk writes are guaranteed atomic by the disk hardware.

Because the root node address has either the old or new value (never a
bit of old and a bit of new), other processes can read from the database
without getting a lock. An external process might see the old or the new
tree, but never a mix of the two. In this way, commits are atomic.

Because we write the new data to disk and call the \texttt{fsync}
syscall\footnote{Calling \texttt{fsync} on a file descriptor asks the
  operating system and hard drive (or SSD) to write all buffered data
  immediately. Operating systems and drives don't usually write
  everything immediately in order to improve performance.} before we
write the root node address, uncommitted data are unreachable.
Conversely, once the root node address has been updated, we know that
all the data it references are also on disk. In this way, commits are
also durable.

We're done!

\aosasectii{How NodeRefs Save Memory}\label{how-noderefs-save-memory}

To avoid keeping the entire tree structure in memory at the same time,
when a logical node is read in from disk the disk address of its left
and right children (as well as its value) are loaded into memory.
Accessing children and their values requires one extra function call to
\texttt{NodeRef.get()} to dereference (``really get'') the data.

All we need to construct a \texttt{NodeRef} is an address:

\begin{verbatim}
+---------+
| NodeRef |
| ------- |
| addr=3  |
| get()   |
+---------+
\end{verbatim}

Calling \texttt{get()} on it will return the concrete node, along with
that node's references as \texttt{NodeRef}s:

\begin{verbatim}
+---------+     +---------+     +---------+
| NodeRef |     | Node    |     | NodeRef |
| ------- |     | ------- | +-> | ------- |
| addr=3  |     | key=A   | |   | addr=1  |
| get() ------> | value=B | |   +---------+
+---------+     | left  ----+
                | right ----+   +---------+
                +---------+ |   | NodeRef |
                            +-> | ------- |
                                | addr=2  |
                                +---------+
\end{verbatim}

When changes to the tree are not committed, they exist in memory with
references from the root down to the changed leaves. The changes aren't
saved to disk yet, so the changed nodes contain concrete keys and values
and no disk addresses. The process doing the writing can see uncommitted
changes and can make more changes before issuing a commit, because
\texttt{NodeRef.get()} will return the uncommitted value if it has one;
there is no difference between committed and uncommitted data when
accessed through the API. All the updates will appear atomically to
other readers because changes aren't visible until the new root node
address is written to disk. Concurrent updates are blocked by a lockfile
on disk. The lock is acquired on first update, and released after
commit.

\aosasectii{Exercises for the Reader}\label{exercises-for-the-reader}

DBDB allows many processes to read the same database at once without
blocking; the tradeoff is that readers can sometimes retrieve stale
data. What if we needed to be able to read some data consistently? A
common use case is reading a value and then updating it based on that
value. How would you write a method on \texttt{DBDB} to do this? What
tradeoffs would you have to incur to provide this functionality?

The algorithm used to update the data store can be completely changed
out by replacing the string \texttt{BinaryTree} in
\texttt{interface.py}. Data stores tend to use more complex types of
search trees such as B-trees, B+ trees, and others to improve the
performance. While a balanced binary tree (and this one isn't) needs to
do $O(log_2(n))$ random node reads to find a value, a B+ tree needs many
fewer, for example $O(log_{32}(n))$ because each node splits 32 ways
instead of just 2. This makes a huge different in practice, since
looking through 4 billion entries would go from $log_2(2^{32}) = 32$ to
$log_{32}(2^{32}) \approx 6.4$ lookups. Each lookup is a random access,
which is incredibly expensive for hard disks with spinning platters.
SSDs help with the latency, but the savings in I/O still stand.

By default, values are stored by \texttt{ValueRef} which expects bytes
as values (to be passed directly to \texttt{Storage}). The binary tree
nodes themselves are just a sublcass of \texttt{ValueRef}. Storing
richer data (via \href{http://json.org}{\texttt{json}} or
\href{http://msgpack.org}{\texttt{msgpack}}) is a matter of writing your
own and setting it as the \texttt{value\_ref\_class}.
\texttt{BinaryNodeRef} is an example of using
\href{https://docs.python.org/3.4/library/pickle.html}{\texttt{pickle}}
to serialise data.

Database compaction is another interesting exercise. Compacting can be
done via an infix-of-median traversal of the tree writing things out as
you go. It's probably best if the tree nodes all go together, since
they're what's traversed to find any piece of data. Packing as many
intermediate nodes as possible into a disk sector should improve read
performance, at least right after compaction. There are some subtleties
here (for example, memory usage) if you try to implement this yourself.
And remember: always benchmark performance enhancements before and
after! You'll often be surprised by the results.

\aosasectii{Patterns and Principles}\label{patterns-and-principles}

Test interfaces, not implementation. As part of developing DBDB, I wrote
a number of tests that described how I wanted to be able to use it. The
first tests ran against an in-memory version of the database, then I
extended DBDB to persist to disk, and even later added the concept of
NodeRefs. Most of the tests didn't have to change, which gave me
confidence that things were still working as expected.

Respect the Single Responsibility Principle. Classes should have at most
one reason to change. That's not strictly the case with DBDB, but there
are multiple avenues of extension with only localised changes required.
Refactoring as I added features was a pleasure!

\aosasectii{Summary}\label{summary}

DBDB is a simple database that makes simple guarantees, and yet things
still became complicated in a hurry. The most important thing I did to
manage this complexity was to implement an ostensibly mutable object
with an immutable data structure. I encourage you to consider this
technique the next time you find yourself in the middle of a tricky
problem that seems to have more edge cases than you can keep track of.

\end{aosachapter}
